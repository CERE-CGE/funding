
\documentclass[10pt,a4paper]{article}

\usepackage[T1]{fontenc}
\usepackage[utf8]{inputenc}
\usepackage[swedish]{babel}
\usepackage{comment}
\usepackage[round]{natbib}
\usepackage{amsmath}
\usepackage{amsfonts}
\usepackage{amssymb}
\usepackage{graphicx}
\usepackage{rotating}
\usepackage{multirow}
\usepackage{multicol}
\usepackage{float}
\usepackage{lscape}
\usepackage{longtable}
\usepackage{threeparttable}
\usepackage{authblk}
\usepackage{url}
\usepackage{booktabs}
\usepackage{subfigure}
\usepackage[margin=1.5in]{geometry}
\usepackage{placeins}         % Permets de vider le buffer de flottants avec \FloatBarrier
\usepackage{epstopdf}
\usepackage{pgfgantt}


\usepackage{color,framed}            % Utilisation des couleurs et de l'environnement shaded
\definecolor{shadecolor2}{rgb}{.92,.92,.92} % choix de la teinte de ``shaded''
\definecolor{shadecolor}{rgb}{.6,.95,.6} % choix de la teinte de ``shaded''
\usepackage[pagebackref=true,
            colorlinks=true,
            linkcolor=blue,
            anchorcolor=blue,
            citecolor=blue,
            filecolor=blue,
            menucolor=blue,
            urlcolor=blue,
            plainpages=true,
            pdfpagemode=UseThumbs,
            pdftitle={Titre},
            pdfauthor={ORDECSYS},
            pdfsubject={Sujet},
            pdfstartview=FitH]{hyperref} % Extensions PDF
\def\pdfBorderAttrs{/Border [0 0 0] } % Options PDF (No border around Links)
\renewcommand\Affilfont{\itshape\small}
\linespread{1.5}


\begin{document}
\textbf{Publication list Tommy Lundgren}\\*
\textbf{Peer-reviewed articles}\\*
\begin{enumerate}
	\item Kriström, B. and T. Lundgren (2005), ”Swedish CO2-Emissions 1900-2010: An Exploratory Note”, Energy Policy 33(9), June 2005, 1223 – 1230. 
	\item Lundgren, T., (2005), “Assessing the Investment Performance of Swedish Timberland: A CAPM Approach”, Land Economics 81(3), August, 353 - 362. 
	\item Lundgren, T., C. Sandström, T. Willebrand  (2006) ”Reaching for new perspectives on socio-ecological systems: Exploring the possibilities for adaptive co-mangagement in the Swedish mountain region”, The International Journal of Biodiversity Science and Management 2(4), 359-70. 
	\item Brännlund, R. and T. Lundgren (2007) “Swedish Industry and Kyoto – An Assessment of the Effects of the European CO2 Emission Permit Trading System”, Energy Policy 35(9), 4749-4762.
	\item Hammar, H., T. Lundgren, M. Sjöström (2008) ”The significance of transport costs in Swedish forest industry”, Journal of Transport Economics and Policy 42(1), 83-104.  
	\item T. Lundgren, Per-Olov Marklund, Brännlund, R., B. Kriström, (2008/2009), ”The Economics of Biofuels”,  International Review of Environmental and Resource Economics, Vol 2, 237-280.  
	\item Lundgren, T. (2009) ”Environmental Protection and Impact on Adjacent Economies”, Growth and Change 40(3), 513-532. 
	\item Brännlund, R., T. Lundgren (2009) “Environmental policy without costs? A review of the Porter hypothesis”, International Review of Environmental and Resource Economics 3, 1-43. 
	\item Bostedt, G., T. Lundgren (2010) “Accounting for Cultural Heritage - A Theoretical and Empirical Exploration with Focus on Swedish Reindeer Husbandry”, Ecological Economics 69(3), 651-657.
	\item Lundgren, T., R. Olsson (2009) “How bad is bad news? Assessing the effects of environmental incidents on firm value”, American Journal of Finance and Accounting 1(4), 376-92.  
	\item Brännlund, R., T. Lundgren (2010), “Environmental Policy and Profitability – Evidence from Swedish Industry”, Environmental Economics and Policy Studies 12(1-2), 59-78.  
	\item Lundgren, T., R. Olsson (2010). “Environmental Incidents and Firm Value - International Evidence using a Multi-Factor Event Study Framework”, Applied Financial Economics 20(16), 1293 - 1307. 
	\item Lundgren, T. (2011), “A micro-economic model of corporate social responsibility”, Metroeconomica  62(1), 69-95. 
	\item Hammar, H., T. Lundgren, M. Sjöström, M. Andersson (2011), “The Kilometer Tax and Swedish Industry”, Applied Economics 43, 2907-2917. 
	\item Fangmiao, H., F. Yi-zhong, R. Brännlund, T. Lundgren (2011). Implications of European low-carbon energy policy changes for the Swedish and Global forest products sectors: An analysis based on GFPM. E-Business and E-Government (ICEE) International Conference Publications. 
	\item Lundgren, T., P-O. Marklund (2012), ”Bioenergy and carbon neutrality”, Journal of Forest Economics 18(1), 91-93.  
	\item Brännlund, R., O. Carlén, Lundgren, T., P-O Marklund (2012). ”The costs and benefits of intensive forest management”, Journal of Benefit-Cost Analysis 3(4). 
	\item Lundgren, T., P-O Marklund (2013), “Biofuel economics – A review”, in print Encyclopedia of Energy, Natural Resource and Environmental Economics, Elsevier, 
London, UK. Ed. J. Shogren. 
	\item Lundgren, T., P-O. Marklund (2013), “Assessing the welfare effects of promoting biomass growth and the use of bioenergy”, in press Climate Change Economics. 
\end{enumerate}
 
\textbf{Books, book chapters}\\*
\begin{enumerate}
	\item Osäkrat klimat – laddad utmaning, Sep 2009, Formas Fokuserar-book, chapter on CSR and welfare (co-authored with P. Cerin). Also tranlated to english in 2010, ”Climate challenge – the saftey’s off”. 
	\item “Hållbar utveckling - Från risk till värde” (2011). Studentlitteratur. Eds. L. Hassel, L-O Larsson, E. Nore. One chapter on sustainable development in Swedish industry. 
	\item Lundgren, T., J. Stage, T. Tangerås (2013). Energimarknaden, ägande och klimatet. SNS förlag: Stockholm (in press). 
\end{enumerate}

\textbf{Unpuplished working papers (available on web)}\\*
\begin{enumerate}
	\item Lundgren, T. (2005) “Effekter på basindustrin av förändringar i energiskattesystemet – simulering med en faktorefterfrågemodell” Working paper 351, Department of Forest Economics, SLU, Umeå.
	\item Lundgren, T. (2005) ”Measuring Regional Welfare Considering Natural and Cultural Resources”, Economics Discussion Papers No. 0510, Department of Economics, University of Otago, Dunedin, New Zealand.  
	\item Lundgren, T. (2006) ”Att mäta regional välfärd och ta hänsyn till natur- och kulturkapital”, Working paper 358, Department of Forest Economics, SLU, Umeå. 
	\item Lundgren, T. (2007) “On the economics of corporate social responsibility”, Working paper series Socially Investment Research Platform, SIRP WP 07-03 (www.sirp.se) 
	\item Brännlund, R., T. Lundgren (2008) “Environmental Policy and Profitability”, UES 750, Dept of Economics, Umeå University. 
	\item Lundgren, T., P-O Marklund (2010) “Climate Policy and Profit Efficiency”, WP 11, Centre for Environmental and Resource Economics, www.cere.se.
	\item Brännlund, R., Lundgren, T., P-O Marklund (2011) “Environmental performance and climate policy”, WP 06-2011, Centre for Environmental and Resource Economics, www.cere.se.  
	\item Brännlund, R and Lundgren, T. Effekter för den elintensiva industrin av att dessa branscher i olika grad omfattas av kvotplikt inom elcerifikatsystemet WP 07-2011, Centre for Environmental and Resource Economics, www.cere.se.  
	\item Jaraite, J., Kazukauskas, A. and Lundgren, T. (2012) “Determinants of Environmental Expenditure and Investment: Evidence from Sweden,” WP 07-2012, Centre for Environmental and Resource Economics, www.cere.se. 
	\item Lundgren, T., P-O. Marklund (2012), “Environmental performance and profits” WP 08-2012, Centre for Environmental and Resource Economics, www.cere.se.* 
	\item Färe, R., S. Grosskopf, T. Lundgren, P-O. Marklund and W. Zhou (2012) “Productivity: Should we Include Bads? Centre for Environmental and Resource Economics”, CERE Working Paper 2012:13, www.cere.se. 
	\item Färe, R., S. Grosskopf, T. Lundgren, P-O. Marklund and W. Zhou (2012) “Pollution generating technologies and environmental efficiency”. Centre for Environmental and Resource Economics, CERE Working Paper 2012:16, www.cere.se.* 
\end{enumerate}

\textbf{Popular science articles}\\*
\begin{enumerate}
	\item “Hållbar utveckling - Från risk till värde” (2011). Studentlitteratur. Eds. L. Hassel, L-O Larsson, E. Nore. One chapter on sustainable development in Swedish industry. 
	\item Lundgren, T. (2002) “Miljöinvesteringar lönar sig”, Fakta-Skog 1, 2002. 
	\item Lundgren, T. (2004) ”Vad kostar en offensiv klimatpolitik?”, Ekonomisk Debatt 6, 19-32. 
	\item Lundgren, T. (2005) “ Varför investerar en del företag frivilligt i miljövänlig teknologi?”, prize essay Thule Yearbook 2005, ISSN 0280-8692, Royal Skytteanska Society. 
	\item Västerbottenskuriren (VK), 18/7, 2009, ”En ohelig allians”. English translation: ”Unholy alliance in climate policy”. (co-authored with R. Brännlund and B. Kriström) 
	\item “För eller emot – Har E85 en framtid som miljöbränsle?”. Comment on the issue in Miljörapporten Nr 5/2011. Published commentary on “Ska jag tanka etanol? (Semida Silveira) FORES Think Tank Study 2011:1, Essays on Environmental Economics and Entreprenurship. 
	\item Lundgren, T. and P-O. Marklund (2012) Bioenergi, klimat och Ekonomi, Miljöforskning – Formas tidning för ett hållbart samhälle, No 1, March 2012. 
\end{enumerate}
 

\textbf{Publication list Örjan Furtenback}
\textbf{Peer-reviewed articles}
\begin{enumerate}
	\item Furtenback, Ö. (2008) “Demand for waste as fuel in the swedish district heating sector: A production function approach”, Waste Management 29(1), 285-292.
\end{enumerate}

\textbf{Unpuplished working papers (available on web)}
\begin{enumerate}
	\item Furtenback, Ö. (2009) “Towards a Functional Ecol-Econ CGE Model with a Forest as Biomass Capital”, EcoMod Press, conference proceedings 2009.
	\item Furtenback, Ö.  “Dynamic CGE-model with heterogeneous forest biomass:Applications to climate policy”, CERE WP 10/2011 (2011).
\end{enumerate}

\textbf{Reports in Swedish}
\begin{enumerate}
	\item Konjunkturinstitutet (2012), ”Miljö, ekonomi och politik 2012”, Konjunkturinstitutet, Stockholm.
	\item Konjunkturinstitutet (2013), ”Miljö, ekonomi och politik 2013”, Konjunkturinstitutet, Stockholm.
\end{enumerate}
 

\end{document}

