
\documentclass[10pt,a4paper]{article}

\usepackage[round]{natbib}
\usepackage{amsmath}
\usepackage{amsfonts}
\usepackage{amssymb}
\usepackage{graphicx}
\usepackage{rotating}
\usepackage{multirow}
\usepackage{multicol}
\usepackage{float}
\usepackage{lscape}
\usepackage{longtable}
\usepackage{threeparttable}
\usepackage{authblk}
\usepackage{url}
\usepackage{booktabs}
\usepackage{subfigure}
\usepackage[margin=1.5in]{geometry}
\usepackage{placeins}         % Permets de vider le buffer de flottants avec \FloatBarrier
\usepackage{epstopdf}

\usepackage{color,framed}            % Utilisation des couleurs et de l'environnement shaded
\definecolor{shadecolor2}{rgb}{.92,.92,.92} % choix de la teinte de ``shaded''
\definecolor{shadecolor}{rgb}{.6,.95,.6} % choix de la teinte de ``shaded''
\usepackage[pagebackref=true,
            colorlinks=true,
            linkcolor=blue,
            anchorcolor=blue,
            citecolor=blue,
            filecolor=blue,
            menucolor=blue,
            urlcolor=blue,
            plainpages=true,
            pdfpagemode=UseThumbs,
            pdftitle={Titre},
            pdfauthor={ORDECSYS},
            pdfsubject={Sujet},
            pdfstartview=FitH]{hyperref} % Extensions PDF
\def\pdfBorderAttrs{/Border [0 0 0] } % Options PDF (No border around Links)
\renewcommand\Affilfont{\itshape\small}
\linespread{1.5}


\begin{document}

\title{Nuclear Phase out in Sweden}
\date{}
\author{}
\maketitle

\section{Summary}
\section{Popular scientific description (in Swedish)}
\section{Total project budget}
\section{Research programme (Appendix A)}
\subsection{Purpose and aims}
\subsection{Survey of the field}
Nuclear phase out issue at global and European level.
Emission targets:Swedish obligation under burden sharing agreement.
The future of nuclear power is considerably uncertain in different countries due to economic, technological and political factors \citep{Joskow2012}.
On one hand, nuclear power continues to generate enthusiasm based on its potential to reduce greenhouse gas emissions and comparatively cheaper to renewable energy \citep{Davis2012, Renssen2013}. On the other hand, there is growing opposition against nuclear power regarding its safety measures, handling and storage of spent fuel, proliferation of nuclear weapons. Germany and Switzerland have already decided to phase out the nuclear power completely from their energy portfolio by  2022 and 2034 respectively. The position taken by Germany and Switzerland points to the potential tension between a need for climate mitigation and a desire for a society free from nuclear power \citep{Glomsrod2013}. It is quite likely that Sweden faces similar challenges in coming future whether to phase out nuclear power or not with emission targets for 2030. In this current context, it is important to understand what could be an economic implication of nuclear phase out for Sweden. How this will impact the electricity market and emission in Sweden ? Can Sweden afford to phase out Nuclear power with potential increase in hydropower generation due to climate change?
Nuclear safety and waste disposal.: Number of people surrounding nuclear power plant at 30 km radius.
Current political situation regarding nuclear phase out in Sweden: What current government says and what its opposition parties says
To represent the government plan,
To represent the opposition parties plan,
Potential role of hydropower, renewable and carbon capture and sequestration.
Current energy status of Sweden.
Considering the nuclear power to be phase out, and the possibility of substituting it by relatively in expensive technique of fossil fuel power is excluded due to CO2 commitment and hydropower development is restricted due to its capacity, then the price and supply of electricity will most certainly be affected.

The purpose of this research is to examine the effects of different policy scenarios with respect to Swedish energy policy, specifically issues concerning a nuclear phase-out and restrictions on CO2 emissions.


\subsection{Project description}
\subsubsection{Methodological framework}
\textbf{General equilibrium model}
Dynamic (recursive) general equilibrium modeling
- Martin Hill model (?)
- Sweden
- sectors(??)
- timeframe (2050)

\textbf{Modeling electricity sector}
Disaggregated electricity sector
- Coal, Gas, Oil, Nuclear, Hydro, Wind
- to incorporate natural resource constraints on hydro and wind, the expansions of wind and hydro power are bounded by the levels of hydro and wind resource factors, respectively.

\textbf{Scenarios}
- Scenario 1: Phase out of existing nuclear power (without expansion of hydro power plant)
- Scenario 2: Phase out of existing nuclear power plant (with expansion of hydro power plant)
- Scenario 3: Phase out of existing nuclear power plant (without expansion of hydro power plant) and CO2 commitment
- Scenario 4: Phase out of existing nuclear power plant (with expansion of hydro power plant) and CO2 commitment


\subsection{Significance}

\subsection{Preliminary results}


\section{CV (Appendix B)}
\section{Publication list (Appendix C)}
\section{Budget and research resources (Appendix N)}
\section{Signatures (Appendix S)}



\pagebreak
\addcontentsline{toc}{chapter}{Bibliography}
\bibliography{RefAbstracts}
\bibliographystyle{plainnat}

\end{document}