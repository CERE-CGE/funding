\documentclass[11pt,a4paper]{extarticle}
\usepackage[T1]{fontenc}
\usepackage[utf8]{inputenc}
\usepackage[swedish]{babel}
\usepackage{comment}
\usepackage[round]{natbib}
\usepackage{amsmath}
\usepackage{amsfonts}
\usepackage{amssymb}
\usepackage{graphicx}
\usepackage{rotating}
\usepackage{multirow}
\usepackage{multicol}
\usepackage{float}
\usepackage{lscape}
\usepackage{longtable}
\usepackage{threeparttable}
\usepackage{authblk}
\usepackage{url}
\usepackage{booktabs}
\usepackage{subfigure}
\usepackage[margin= 25 mm]{geometry}
\usepackage{placeins}         % Permets de vider le buffer de flottants avec \FloatBarrier
\usepackage{pgfgantt}
\renewcommand{\rmdefault}{phv} % Arial
\renewcommand{\sfdefault}{phv} % Arial


\usepackage{color,framed}            % Utilisation des couleurs et de l'environnement shaded
\definecolor{shadecolor2}{rgb}{.92,.92,.92} % choix de la teinte de ``shaded''
\definecolor{shadecolor}{rgb}{.6,.95,.6} % choix de la teinte de ``shaded''
\usepackage[pagebackref=true,
            colorlinks=true,
            linkcolor=blue,
            anchorcolor=blue,
            citecolor=blue,
            filecolor=blue,
            menucolor=blue,
            urlcolor=blue,
            plainpages=true,
            pdfpagemode=UseThumbs,
            pdftitle={Titre},
            pdfauthor={ORDECSYS},
            pdfsubject={Sujet},
            pdfstartview=FitH]{hyperref} % Extensions PDF
\def\pdfBorderAttrs{/Border [0 0 0] } % Options PDF (No border around Links)
\renewcommand\Affilfont{\itshape\small}
\linespread{1.5}

% Times New Roman with size 12pt recommended
\begin{document}

\title{Assessing nuclear phase out in Sweden from an energy and macroeconomic perspective}
\maketitle
\author{}
\date{}
\section{Summary}
Sweden has the ambition of showing the way to a sustainable society with many implications for the energy system and it is necessary to invest in economics research to better understand the impacts of such ambitious energy policies. There is political consensus on the expansion of renewable energy, promotion of energy efficiency and non expansion of hydropower. Nonetheless, the main political conflict regards the future of nuclear power. A restriction in the supply of nuclear power will have significant economic consequences in a country like Sweden with a large element of power-intensive industries dependent on a stable and affordable supply of electricity. Given the ambitious GHGs emission reduction target, and the challenge of meeting the increasing demand of electricity, Swedish nuclear power policy and electricity sector needs to be scrutinized further; the economic stakes are considerable. The research project we are proposing here intends to analyze the socio-economic costs and effects of restrictions on nuclear power capacity in combination with an expansion of renewable weather dependent or intermittent power sources, in Swedish electricity production. With the type of CGE model we propose, it will be possible to analyze the effects of policy measures targeted at singular as well as groups of power sources in a system perspective. The tool will be an important contribution to the analysis ability of Swedish energy policy in general and policies targeting phase out of nuclear power in particular.

Sweden seeks to be a forerunner in climate policy and governmental reports stress the urgency of curbing anthropogenic atmospheric greenhouse gases (Roadmap for a fossil fuel independent transport system in 2030 \cite{SOU201384} and Sweden - an emissions-neutral country by 2050 \cite{sepa2012}). There is also political consensus on the expansion of renewable energy, promotion of energy efficiency and non expansion of hydropower. Nonetheless, the main political conflict regards the future of nuclear power. The green party wants a quick phase out, while e.g. the liberal party support replacement of existing reactors. A larger variability and increasing demand for electricity is expected \cite{sepa2012, SOU201384}. Given the ambitious GHGs emission reduction target \cite{SOU201384}, and the challenge of meeting the increasing demand of electricity, Swedish nuclear power policy and electricity sector needs to be scrutinized further; the economic stakes are considerable.


\section{Popular scientific description (in Swedish)}
Det råder parlamentarisk oenighet i frågan om kärnkraftens vara eller icke vara i den svenska energipolitiken. En begränsning i utbudet av kärnkraften kommer att få ekonomiska konsekvenser i ett land som Sverige med ett stort inslag av elintensiv industri som är beroende av en stabil elförsörjning. Det forskningsprojektet vi föreslår avser att analysera samhällsekonomiska konsekvenser och spridningseffekter av restriktioner på svensk kärnkraftskapacitet i kombination med en utbyggnad av förnybar elproduktion. För att åstadkomma detta kommer vi att konstruera en allmänjämnviktsmodell med fokus på energimarknaderna. Allmänjämviktsmodellers förmåga att beskriva ekonomiövergripande effekter utgör en viktig skillnad jämfört med andra så kallade partiella modeller. En allmänjämviktsmodell kan fånga upp de återverkningar som sker mellan olika sektorer vid förändringar av ekonomiska förutsättningar och inte bara den direkta påverkan i de berörda sektorerna. Jämfört med partiella modeller fångas de totala samhällsekonomiska konsekvenserna upp på ett mer fullständigt sätt i allmänjämviktsmodeller. Med den typ av allmänjämviktsmodell som vi föreslår, är det möjligt att analysera effekterna av politiska åtgärder riktade till såväl individuella som grupper av kraftkällor i ett systemperspektiv. Verktyget kommer att vara ett viktigt bidrag till förmågan att analysera den svenska energipolitiken i allmänhet och åtgärder som riktar sig utfasning av kärnkraften i synnerhet.
\section{Total project budget}
\section{Research programme (Appendix A)}
\textbf{Purpose and aims}\\
The research project we are proposing here intends to analyze the socio-economic costs and effects of any restrictions on the capacity of the nuclear power combined with an increased demand for balancing power, due to plans of expansion of renewable weather-dependent power sources. The goal is to analyze different scenarios under which the Swedish nuclear power system might develop over the next couple of decades. To this end, an existing CGE model is targeted for enhancement on detailed description of the Swedish energy sector. This type of model, called a hybrid model, is a combination of detailed micro data on the energy sector and more aggregated data usually used in computable general equilibrium (CGE) models. This hybrid model describes the energy sector in sufficient detail enabling analysis of changes in the economic conditions of different power sources for electricity generation. This type of CGE model is currently missing for Sweden and would be a genuine contribution to the analysis of energy policy in general, but also a particularly well adapted tool for the project we are proposing here.

Energy issues are becoming increasingly important and related considerations increasingly complex. Surprisingly, energy economics is to a large extent a neglected field at Swedish universities, at least at the economics departments. CERE has launched a significant effort in this subject field. An important reason is the need for development of knowledge in CGE modeling in Sweden with focus on energy issues. Seeing the energy policy developments in a system perspective also seems increasingly important, especially in view of what has happened in the energy markets today.

Water for hydropower can be stored in reservoirs and production can hence efficiently adjust to the demand for electricity or variation of another generation, which makes it very important in the Swedish energy system. It is especially important in an energy system that is including more renewable intermittent sources, such as solar and wind power. Hydropower also has a function as base load\footnote{The minimum amount of electric power delivered or required over a given period at a constant rate.} in the Swedish electricity mix due to its large share of the Swedish electricity production mix. Another important component in the Swedish electricity production system is nuclear power. Nuclear power function as provider of base load power in Sweden but is not as important as hydropower in terms of balancing fluctuating electricity demand. A reduction in the base load capacity by reducing or decommission nuclear power while at the same time increasing intermittent sources will put a stress on hydropower in terms of both increased base load requirement and increased balancing demand at the same time. This will most certainly have impact on the scale as well as the stability of the Swedish electricity system.

There is some consensus among the leading parliamentary parties on electricity policies regarding hydropower, renewable energy, and opportunities to work with energy efficiency. The great dividing line and the main conflict lies in the perception of nuclear power. The spectra lies between those who want a quick phase out, and those who are willing to replace existing reactors.

An important question in this political setting is to what degree the Swedish economy will adapt to potential major changes in the electricity production system. A restriction in the supply of nuclear will most likely have major economic consequences in a country like Sweden with a large element of power-intensive industries dependent on stable supply of electricity. Furthermore, it is important to try to assess the spillover effects of such restrictions on the wider economy. This requires, in principle, detailed knowledge of every market in the country's economy and how these markets interact. A typical economic model to analyze this type of interaction between markets is a general equilibrium model.

CERE has recently started a group (CGE Group) to specifically analyze the Swedish and European environmental and energy policies with the help of general equilibrium modeling (CGE). CGE models' ability to describe the economy-wide effects is an important difference compared with different types of partial models. A CGE model can capture the repercussions that occur between different sectors and not only the direct impact in the sectors concerned. Compared to partial models, CGE models captures the overall socio-economic impact in a more complete and comprehensive way. The model we propose will be built upon a model (LUGEM08) constructed for specific use in the Long-Term Survey 2008, and has also been used for a number of studies of Swedish climate policy.

With the type of CGE model we propose, it will be possible to analyze the effects of policy measures targeted at individuals as well as groups of power sources in a system perspective. The tool will be an important contribution to the analysis of Swedish energy policy.

\textbf{Survey of the field}\\
Globally, nuclear energy contribute around 14\% for the world electricity generation \citep{OECD2012}. Many countries in Europe depend on nuclear energy for the electricity generation with 75\% in France , 54\% in Slovak Republic , 51\% in Belgium\footnote{http://www.iaea.org/PRIS/WorldStatistics/NuclearShareofElectricityGeneration.aspx}. About 40\% of total Swedish electricity generation in 2011 was produced from nuclear energy \citep{SEA2012}. However, the future of nuclear power is considerably uncertain in different countries due to economic, technological, environmental and political factors \citep{Joskow2012}. On one hand, nuclear power continues to generate enthusiasm based on its potential to reduce greenhouse gas emissions and comparatively cheaper to renewable energy \citep{Davis2012, Renssen2013}. On the other hand, there is growing opposition against nuclear power regarding its safety measures, handling and storage of spent fuel, and proliferation of nuclear weapons. After the Fukushima nuclear disaster, Germany and Switzerland have decided to phase out the nuclear power completely from their energy portfolio by  2022 and 2034 respectively. The position taken by Germany and Switzerland points to the potential tension between a need for climate mitigation and a desire for a society free from nuclear power \citep{Glomsrod2013}. It is quite likely that Sweden faces similar challenges in the coming future that is whether to phase out nuclear power or not with ambitious emission targets for 2030. Without a well defined capacity replacement plan, the countries could face a severe shortage of electricity supply and hence significant cost to the society. In this current context, it is important to understand what the economic implications of a nuclear phase out for Sweden, and how will this impact the electricity market and emissions in Sweden?

Several studies have shown the potential economic consequences of nuclear phase-out \citep{Bohringer2002, Nestle2012, Bretschger2012, Duscha0, Glomsrod2013, Kunsch2014}. The study by \cite{Bretschger2012} for Switzerland case showed that the nuclear phase-out can be achieved at moderate costs by investing in innovative industries and through structural change to less energy dependent economy. Moreover, if trade options within European countries are unrestricted the the short or medium term effect on electricity price will be moderate \citep{Glomsrod2013}. In a study \citep{Chen2013} of Taiwan, it is shown that it is crucial to convert industrial structure into less energy intensive if country wants to achieve non-nuclear and low carbon environment. Unlike many other studies where it is shown that domestic electricity price to decrease due to extending nuclear plant life or commissioning new nuclear plant in other countries, the study by \citep{Nestle2012} presents the evidence that this is unlikely to happen in Germany. Nuclear energy policy study on Belgium by \cite{Kunsch2014} found that a too early nuclear phase out means that Belgium economy has to rely on foreign suppliers and increased use of non-renewable and CO$_2$ emitting fossil fuels.

There are only few studies done of this in Swedish context. The study by \cite{Bergman1981} used a general equilibrium model and showed that given that the product and factor markets functions smoothly, Sweden can accommodate quite significant changes in electricity supply conditions without major changes in the main economic indicators. However, adjustment costs (reallocation of labor force, transportation cost, capital losses etc.) which are neglected in the model could give quite different results. The study by \cite{Andersson1997} used a partial equilibrium model and showed that phasing out nuclear power while restricting future CO$_2$ emissions to the 1990 level implies a significant increase in electricity prices and a substantial loss in welfare. In the Swedish Energy and Environment Policy (SEEP) model, \cite{nordhaus1997swedish} estimates the cost of nuclear phase out in Sweden (from 2000 to 2010) to 5 percent of 2010 GDP in the Swedish context. This model though, is somewhat dated and lacks a detailed description of the Swedish economy and is thus not able to capture the repercussions that occur between different sectors. Furthermore, it is important to study the current ongoing debate on nuclear phase-out in the Swedish energy and economic context with current energy and climate policy objectives.

\textbf{Project description}\\
%Summarise the project. Describe theories, methods, timetable, implementation and project organisation.
\textbf{Theories and methods}\\
As mentioned in the section Purpose and aims, a computable general equilibrium (CGE) model will be used for the purpose and aim of our research. A CGE model is theoretically consistent with microeconomic foundations. In such a model, each sector and consumer optimizes their production and consumption decision at different stages of production and utility tree within their resource, technology and budgetary constraints, respectively. These models also ensure the balance of income and expenditure of the economic agents, market clearing conditions (supply equals demand) and zero profit conditions (sum of inputs is equal to output) for each sector (commodity). Moreover, it can simulate effects not just parts of the economy that are of particular interest, but also of the effects on the entire economy. Hence, a CGE model becomes appropriate in this research when a nuclear phase out policy not only affects the electricity sector, but also will have economy-wide effects. These models are also useful for analyzing changes in sectoral output, prices, endowments and trade as well as changes in national efficiency (welfare) consequent to policy changes.

A feature of the Swedish electricity production system is that major parts as hydropower and nuclear power are subject to caps on the production capabilities at least in the short or medium term. The growth of hydropower is restricted and the restriction of the capacity of nuclear power is the target for the analysis. These types of capacity constraints are well suited to be modeled under the mixed complementarity framework \citep{raey}, which we will use for our proposed project.\\

% timetable, implementation and project organisation
% Implementation
% Data
\textbf{Implementation}\\
An initial step for the project is to compile a social accounting matrix (SAM), for Sweden, with specific focus on the energy markets. This work involves the collection of data from different sources and processing of these data so that they can consistently describe the Swedish economy in equilibrium and that the data are of sufficient detail to adequately analyze policies targeting the electricity system. This SAM will feed the model with new and complementary data.
% Model enhancement and scenarios
The existing base model will have to be modified in order to incorporate the new detailed data for the Swedish energy system and calibrated to current economic situation in order to establish a benchmark to compare with relevant policy simulations. Scenarios of different potential policy decisions will be created, and the model will need further modification in order to accommodate simulations of these scenarios.

\textbf{Project organisation}\\
% Project organisation
The research group operates within the Centre for Environmental and Resource Economics, CERE. CERE is an interdisciplinary research centre that unites Umeå University (UmU) and the Swedish University of Agricultural Sciences (SLU). Many prominent researchers are cooperating with CERE comprising an exclusive and important international network. A general aim of CERE is to provide high class research on topics pertaining to energy systems. Of special interest are the effects of climate and energy policy on energy use, which the proposed project aims to study. A well-established and experienced group of researchers can without doubt compose a notable contribution in analyzing and assessing relevant energy issues.\\*
The research group consists of:\\*
\textbf{Professor Tommy Lundgren} (principal investigator), CERE, School of Business and Economics, Umeå University. Professor Lundgren have a solid background in project management and has a number of successfully completed projects related to energy issues (funded by Energimyndigheten, Formas, Handelsbanken, Mistra, etc.).\\*
\textbf{Dr. Örjan Furtenback}, former employee of National Institute for Economic Research as a member of the CGE-group further developing the EMEC model and conducted analysis of the environmental and resource policies on the Swedish economy. As a PhD student, he developed a dynamic CGE-model with age-structured forest resource for analysis of distributional effects of incorporating forest and forest resources in reducing greenhouse gases. Dr. Furtenback will be directly involved with the implementation of the project together with Dr. Joshi.\\*
\textbf{Dr. Santosh Ram Joshi}, a PhD from NUI Galway, Ireland, and a Master from Wageningen University. He has previously worked as a modeler at the Economics and Environmental Laboratory in Switzerland. His research at CERE will in general be focused on computable general equilibrium modeling, and especially on integrating a detailed description of the energy sector into the Swedish model.\\

\textbf{Timetable}\\
% Paper writing (how many)
The research group will in cooperation write papers for peer-reviewed scientific publication and results will be presented at international meetings, workshops and conferences. We will make a special effort to put our results into popular science forums to facilitate the spread of knowledge outside the academic arena.
% Timetable
The most time consuming part of the project will be the compilation of the SAM and model modifications. The compilation of the SAM will be accomplished during the first year. Model modifications will take place during both of the years whereas creation of relevant scenarios for simulations will take place during the second year of the project.\\

\begin{ganttchart}[
hgrid=true,
vgrid= true,
y unit chart=0.5cm,
bar/.style={fill=gray}
]{1}{24}
\gantttitle{Months}{24} \\
\gantttitlelist{1,...,24}{1} \\
\ganttbar{Literature Review}{1}{3} \\
\ganttbar{Database}{3}{12} \\
\ganttbar{Model development}{9}{18} \\
\ganttbar{Model simulation}{16}{22} \\
\ganttbar{Result and analysis}{18}{24} \\
\ganttbar{Final Report}{21}{24}\\
\ganttbar{Meetings}{1}{1}
\ganttbar{}{6}{6}
\ganttbar{}{11}{11}
\ganttbar{}{16}{16}
\ganttbar{}{21}{21}
\end{ganttchart}


%http://ctan.mirrorcatalogs.com/graphics/pgf/contrib/pgfgantt/pgfgantt.pdf

\textbf{Significance}\\
With the ambitious emission targets, and the safety and disposal issue relevant for nuclear power; there is an urgent question whether Sweden should phase out, early phase out and replace its ageing nuclear power, or not. Whatever Sweden decide on all these possible options, it is important to carry out an analysis of its impact on the Swedish economy and the energy system. There are only a few attempts in the past to investigate the socio-economic consequence of nuclear phase out in Sweden. Since then, both the economic and energy systems in Sweden has evolved and changed considerably, and hence this project seeks to advance this research area within the Swedish context. This will be achieved by using a recent database, incorporating a detailed energy and electricity system, and developing relevant scenarios that are required to model the implications of a nuclear phase out. This research will add to the academic discourse on the issue of nuclear phase out and ultimately answer pertinent policy questions to give better guidance to decision and policy making.

\textbf{Preliminary results}\\
Researchers of this project have used CGE models extensively in their research area. The Global Trade Analysis Project (GTAP), a multi-regional, multi-sectoral static general equilibrium model, was used to examine and analyse the complex multi-sectoral and multilateral trade negotiations as are contained in the WTO trade policies. Similarly, GEMINI-E3 (General Equilibrium Model of International-National Interactions between Economy, energy and Environment), a recursive dynamic CGE model, is used to do macroeconomic analyse of the impacts and adaptation to climate change. This model was linked with other systems of models (climate, energy and agriculture) to get wider perspective and implications of climate change. This research was part of Framework Programme 7 project ERMITAGE (Enhancing Robustness and Model Integration for The Assessment of Global Environmental Change). In the Swedish setting the CGE-model EMEC (Environmental Medium term Economic model) of the National institute for economic research in Sweden was used to account for forest sequestration in the analysis of the Swedish vision on carbon dioxide emissions for 2050 as well as the 2030 goal for fossil fuel independence. Also a dynamic CGE-model with age-structured forest resource \citep{furtenback2011three} has been constructed by a member of the team.

\textbf{National and International collaboration}\\
Our research group will benefit from collaboration from some top scientist within the field of CGE modeling that are already affiliated with CERE. The following persons, with specific expertise and responsibility, will act as collaborating members of the project team:\\*
\textbf{Professor Christoph Böhringer}; a world leading authority on Computable General Equilibrium (CGE modeling), is a research associate at ZEW and chair at the Department of Economic Policy, University of Oldenburg, Germany. Professor Böhringer will assist in model development and issues regarding CGE modeling in an energy framework. Professor Böhringer is an important part of the CERE strategy to develop CGE modeling – particularly within the framework of our general work on energy issues. Professor Böhringer is currently working with us to develop a 57-sector GTAP version of a Sweden CGE model, which is much more aggregated regarding the energy sector. Thus, there is a beneficial cross-fertilization between this project and work already ongoing at CERE.\\*
\textbf{Dr. Martin Hill}, employee at the Ministry of Finance and Adjunct Professor at SLU and CERE, is one of Sweden's most skilled CGE specialists. As a collaborating member of the group Dr. Hill’s expertise on Swedish policy issues will be an important input when designing plausible scenarios for current and coming Swedish policies regarding nuclear and renewable power.\\*
\textbf{Dr. Badri Narayanan} is researcher at the Center for Global Trade Analysis, Department of Agricultural Economics, Purdue University, USA. Dr. Narayanan holds an extensive knowledge in data issues regarding CGE modeling and will, as a collaborating member, be an tremendous asset in the labor intense work of constructing the SAM.

\section{CV (Appendix B)}
\textbf{CV Tommy Lundgren}\\*
\textbf{Higher education degree}\\*
1996, Umeå University, Economics.\\*
\textbf{Doctoral degree}\\*
2001, Swedish University of Agricultural Sciences, Economics, “Dynamic Factor Demands and Environmental Investments” (supervised by Bengt Kriström).\\*
\textbf{Post-doctoral positions abroad}\\*
2002, University of Calgary (Canada), Department of Economics.\\*
2005, University of Otago (New Zealand), Department of Economics.\\*
\textbf{Docent level}\\*
2007, Umeå University, Department of Economics, Umeå School of Business and Economics.\\*
\textbf{Current positions}\\*
Main affiliation: Centre for Environmental and Resource Economics (CERE), a joint venture between Umeå University and Swedish University of Agricultural Sciences. (www.cere.se).\\*
1. Professor of Economics (guest), Umeå School of Business and Economics (USBE), Umeå University, Umeå, Sweden. 2012-07-01 - …  . Research 100%.\\*
2. Researcher, Department of Forest Economics, SLU, Umeå, Sweden. 2006-10-01 – … . Research 80\%.\\*
\textbf{Previous positions}\\*
1. Researcher, Department of Forest Economics, SLU, Umeå, Sweden 2001-06-09 – 2006-09-30. Research 80\%.\\*
2. Researcher (FoAss and Lektor), Umeå School of Business and Economics, Umeå University, 2006-10-01 – 2012-06-30. Research 100\%.\\*
\textbf{Supervision}\\*
Ph D’s:\\*
Örjan Furtenback, 2011.\\*
Shanshan Zhang, scheduled 2015.\\*
Mathilda Eriksson, scheduled 2014.\\*
Erik Brockwell, scheduled 2014.\\*
\textbf{Prizes and awards}\\*
Kungliga Skytteanska samfundet (the Royal Skytteanska Society) prize to promising young researcher in environmental economics, May 2004.\\*
The Globe Award (2008). Selected as winner of the Best CSR Research (corporate social responsibility research) “for an outstanding and tangible research in the field of CSR”. The prize was handed out by Her Royal Highness and Crown Princess Victoria Bernadotte.\\*
\textbf{Brief description of past and current projects}\\*
The determinants of investment in the Swedish forest industry, 1996-1998. Wallander/Hedelius fund, Handelsbanken. (with Bengt Kriström).\\*
Environmental investments – applications to the Swedish forest industry, 1998 - 2001. Principal investigator. Wallander/Hedelius fund, Handelsbanken. (with Bengt Kriström).\\*
Swedish timberland investments – assessment of investment performance, 2001 – 2002 (FORMAS). (with Peichen Gong).\\*
The MountainMistra project, 2003-2006. Principal investigator for one of three theoretical frameworks (measuring regional welfare), and also principal researcher in sub-project concerning economic trends and economic development, in the Swedish mountain area.\\*
Taxing waste incineration and the effects of a kilometer tax, 2005 – 2006. Principal investigator. The royal Swedish academy of agriculture and forestry, KSLA.\\*
Direct and indirect effects of environmental policy on the Swedish forest sector, 2006-2008 (FORMAS). Principal investigator (with Bengt Kriström).\\*
The significance of transport costs in Swedish Forest Industry, 2006 – 2007. Effects on factor demand in Swedish manufacturing of a kilometer tax on heavy vehicles. (Joint project with Magnus Sjöström and Henrik Hammar, National institute of economic research, KI).\\*
Sustainable Investment Research Platform (SIRP), 2006-2012. MISTRA financed. Umeå School of Business and Economics (USBE). Principal investigator on sub-project concerning research on corporate social responsibility and environmental performance in Swedish industry. (see www.sirp.se).\\*
Bioenergy, the Climate, and Economics, 2008-2012 (FORMAS). Principal investigator.\\*
The economics of biofuels and the effects on climate and welfare of societies. Impact of energy and environmental policy on sustainable development and competitiveness in Swedish industry, 2011-2013 (Energimyndigheten – the Swedish Energy agency). Principal investigator.\\*

\pagebreak
\textbf{CV Örjan Furtenback}\\*
\textbf{Higher education degree}\\*
1999, Master of Science: Computer Science, The Royal Institute of Technology Stockholm\\*
\textbf{Doctoral degree}\\*
2011, Department of Forest Economics, Swedish University of Agricultural Sciences Umeå, “Three Essays on Swedish Energy and Climate Policy Options -
Dynamic CGE-models with Heterogeneous Forests and an Econometric Model of Fuel
Substitution in District Heating Plants” (supervised by Bengt Kriström and Tommy Lundgren).\\*
\textbf{Post-doctoral positions}\\*
2013, Centre for Environmental and Resource Economics (CERE),  a joint venture between Umeå University and Swedish University of Agricultural Sciences. (www.cere.se).\\*
\textbf{Current positions}\\*
Post-doctoral at the Centre for Environmental and Resource Economics (CERE), research 100 \% 2013-10-01 -.\\*
\textbf{Previous positions}\\*
2012-2013, Researcher at the National Institute of Economic Research, Stockholm, Sweden.\\*
2003, Software consultant at Dotway AB, Malmö.\\*
2002-2003, Self-employed, Stockholm.\\*
1999-2002, Software development manager at SATISFACTORY AB, Stockholm.\\*
1998-1999, Software consultant at Guide Förädling AB, Stockholm.\\*
1997-1998, Software consultant at Resco AB, Stockholm.\\*
1996-1997, Research engineer at Karolinska Institutet, Department of Medical Laboratory Sciences and Technology, Huddinge.\\*

\pagebreak
\textbf{CV Santosh Ram Joshi}\\*
\textbf{Higher education degree}\\*
2005, Master of Science: Environmental Economics, Wageningen University, The Netherland\\*
\textbf{Degree of Doctor}\\*
2011, J.E. Cairnes School of Business and Economics, National University of Ireland Galway, Ireland, Function forms and AGE Models: Understanding the implication of trade liberalization for the Global Economy, supervised by Late Dr Eoghan Garvey, Eithne Murphy and Dr. Hugh Kelley.\\*
\textbf{Postdoctoral position}\\*
2011-2013, Economics and Environmental Management Laboratory, Ecole Polytechnic Federal de Lausanne (EPFL), Switzerland.\\*
\textbf{Present position}\\*
2014 – , Post-Doctoral Researcher, Research 100\%, Centre for Environmental and Resource Economics (CERE) and Umeå School of Business and Economics,Umeå University.\\*
\textbf{Previous position}\\*
2007-2009, Teaching Assistant, J.E. Cairnes School of Business and Economics, National University of Ireland Galway (NUIG), Ireland.\\*
Aug 2004 - Jan 2005, Research Assistant, Canadian Energy Research Institutes (CERI), Canada\\*
2002-2003, Research Assistant, International Union for Conservation of Nature (IUCN),Nepal\\*
\textbf{Other information that is relevant to your application}\\*
\textit{Short course}\\*
August 2007, Purdue University, West Lafayette, Indianapolis, USA on Global Trade Analysis.\\*
September 2010, ETH Zurich, Zurich,  Switzerland on Dynamic General Equilibrium Analysis.\\*
\textit{Scholarships and awards}\\*
TEAGASC Walsh Fellowship awarded by TEAGASC, Ireland to pursue Ph.D.\\*
Netherlands Fellowship Program awarded by Government of Netherlands to pursue Masters of Sciences.\\*
Colombo Plan Scholarship awarded by Government of India to pursue Bachelor of Engineering.\\*
\textit{Visiting Scholar}\\*
Department of Economics, University of California, Davis, USA\\*


\section{Publication list (Appendix C)}
\textbf{Publication list Tommy Lundgren}\\*
\textbf{Peer-reviewed articles}\\*
\begin{enumerate}
	\item Kriström, B. and T. Lundgren (2005), ”Swedish CO2-Emissions 1900-2010: An Exploratory Note”, Energy Policy 33(9), June 2005, 1223 – 1230.
	\item Lundgren, T., (2005), “Assessing the Investment Performance of Swedish Timberland: A CAPM Approach”, Land Economics 81(3), August, 353 - 362.
	\item Lundgren, T., C. Sandström, T. Willebrand  (2006) ”Reaching for new perspectives on socio-ecological systems: Exploring the possibilities for adaptive co-mangagement in the Swedish mountain region”, The International Journal of Biodiversity Science and Management 2(4), 359-70.
	\item Brännlund, R. and T. Lundgren (2007) “Swedish Industry and Kyoto – An Assessment of the Effects of the European CO2 Emission Permit Trading System”, Energy Policy 35(9), 4749-4762.
	\item Hammar, H., T. Lundgren, M. Sjöström (2008) ”The significance of transport costs in Swedish forest industry”, Journal of Transport Economics and Policy 42(1), 83-104.
	\item T. Lundgren, Per-Olov Marklund, Brännlund, R., B. Kriström, (2008/2009), ”The Economics of Biofuels”,  International Review of Environmental and Resource Economics, Vol 2, 237-280.
	\item Lundgren, T. (2009) ”Environmental Protection and Impact on Adjacent Economies”, Growth and Change 40(3), 513-532.
	\item Brännlund, R., T. Lundgren (2009) “Environmental policy without costs? A review of the Porter hypothesis”, International Review of Environmental and Resource Economics 3, 1-43.
	\item Bostedt, G., T. Lundgren (2010) “Accounting for Cultural Heritage - A Theoretical and Empirical Exploration with Focus on Swedish Reindeer Husbandry”, Ecological Economics 69(3), 651-657.
	\item Lundgren, T., R. Olsson (2009) “How bad is bad news? Assessing the effects of environmental incidents on firm value”, American Journal of Finance and Accounting 1(4), 376-92.
	\item Brännlund, R., T. Lundgren (2010), “Environmental Policy and Profitability – Evidence from Swedish Industry”, Environmental Economics and Policy Studies 12(1-2), 59-78.
	\item Lundgren, T., R. Olsson (2010). “Environmental Incidents and Firm Value - International Evidence using a Multi-Factor Event Study Framework”, Applied Financial Economics 20(16), 1293 - 1307.
	\item Lundgren, T. (2011), “A micro-economic model of corporate social responsibility”, Metroeconomica  62(1), 69-95.
	\item Hammar, H., T. Lundgren, M. Sjöström, M. Andersson (2011), “The Kilometer Tax and Swedish Industry”, Applied Economics 43, 2907-2917.
	\item Fangmiao, H., F. Yi-zhong, R. Brännlund, T. Lundgren (2011). Implications of European low-carbon energy policy changes for the Swedish and Global forest products sectors: An analysis based on GFPM. E-Business and E-Government (ICEE) International Conference Publications.
	\item Lundgren, T., P-O. Marklund (2012), ”Bioenergy and carbon neutrality”, Journal of Forest Economics 18(1), 91-93.
	\item Brännlund, R., O. Carlén, Lundgren, T., P-O Marklund (2012). ”The costs and benefits of intensive forest management”, Journal of Benefit-Cost Analysis 3(4).
	\item Lundgren, T., P-O Marklund (2013), “Biofuel economics – A review”, in print Encyclopedia of Energy, Natural Resource and Environmental Economics, Elsevier,
London, UK. Ed. J. Shogren.
	\item Lundgren, T., P-O. Marklund (2013), “Assessing the welfare effects of promoting biomass growth and the use of bioenergy”, in press Climate Change Economics.
\end{enumerate}

\textbf{Books, book chapters}\\*
\begin{enumerate}
	\item Osäkrat klimat – laddad utmaning, Sep 2009, Formas Fokuserar-book, chapter on CSR and welfare (co-authored with P. Cerin). Also tranlated to english in 2010, ”Climate challenge – the saftey’s off”.
	\item “Hållbar utveckling - Från risk till värde” (2011). Studentlitteratur. Eds. L. Hassel, L-O Larsson, E. Nore. One chapter on sustainable development in Swedish industry.
	\item Lundgren, T., J. Stage, T. Tangerås (2013). Energimarknaden, ägande och klimatet. SNS förlag: Stockholm (in press).
\end{enumerate}

\textbf{Unpuplished working papers (available on web)}\\*
\begin{enumerate}
	\item Lundgren, T. (2005) “Effekter på basindustrin av förändringar i energiskattesystemet – simulering med en faktorefterfrågemodell” Working paper 351, Department of Forest Economics, SLU, Umeå.
	\item Lundgren, T. (2005) ”Measuring Regional Welfare Considering Natural and Cultural Resources”, Economics Discussion Papers No. 0510, Department of Economics, University of Otago, Dunedin, New Zealand.
	\item Lundgren, T. (2006) ”Att mäta regional välfärd och ta hänsyn till natur- och kulturkapital”, Working paper 358, Department of Forest Economics, SLU, Umeå.
	\item Lundgren, T. (2007) “On the economics of corporate social responsibility”, Working paper series Socially Investment Research Platform, SIRP WP 07-03 (www.sirp.se)
	\item Brännlund, R., T. Lundgren (2008) “Environmental Policy and Profitability”, UES 750, Dept of Economics, Umeå University.
	\item Lundgren, T., P-O Marklund (2010) “Climate Policy and Profit Efficiency”, WP 11, Centre for Environmental and Resource Economics, www.cere.se.
	\item Brännlund, R., Lundgren, T., P-O Marklund (2011) “Environmental performance and climate policy”, WP 06-2011, Centre for Environmental and Resource Economics, www.cere.se.
	\item Brännlund, R and Lundgren, T. Effekter för den elintensiva industrin av att dessa branscher i olika grad omfattas av kvotplikt inom elcerifikatsystemet WP 07-2011, Centre for Environmental and Resource Economics, www.cere.se.
	\item Jaraite, J., Kazukauskas, A. and Lundgren, T. (2012) “Determinants of Environmental Expenditure and Investment: Evidence from Sweden,” WP 07-2012, Centre for Environmental and Resource Economics, www.cere.se.
	\item Lundgren, T., P-O. Marklund (2012), “Environmental performance and profits” WP 08-2012, Centre for Environmental and Resource Economics, www.cere.se.*
	\item Färe, R., S. Grosskopf, T. Lundgren, P-O. Marklund and W. Zhou (2012) “Productivity: Should we Include Bads? Centre for Environmental and Resource Economics”, CERE Working Paper 2012:13, www.cere.se.
	\item Färe, R., S. Grosskopf, T. Lundgren, P-O. Marklund and W. Zhou (2012) “Pollution generating technologies and environmental efficiency”. Centre for Environmental and Resource Economics, CERE Working Paper 2012:16, www.cere.se.*
\end{enumerate}

\textbf{Popular science articles}\\*
\begin{enumerate}
	\item “Hållbar utveckling - Från risk till värde” (2011). Studentlitteratur. Eds. L. Hassel, L-O Larsson, E. Nore. One chapter on sustainable development in Swedish industry.
	\item Lundgren, T. (2002) “Miljöinvesteringar lönar sig”, Fakta-Skog 1, 2002.
	\item Lundgren, T. (2004) ”Vad kostar en offensiv klimatpolitik?”, Ekonomisk Debatt 6, 19-32.
	\item Lundgren, T. (2005) “ Varför investerar en del företag frivilligt i miljövänlig teknologi?”, prize essay Thule Yearbook 2005, ISSN 0280-8692, Royal Skytteanska Society.
	\item Västerbottenskuriren (VK), 18/7, 2009, ”En ohelig allians”. English translation: ”Unholy alliance in climate policy”. (co-authored with R. Brännlund and B. Kriström)
	\item “För eller emot – Har E85 en framtid som miljöbränsle?”. Comment on the issue in Miljörapporten Nr 5/2011. Published commentary on “Ska jag tanka etanol? (Semida Silveira) FORES Think Tank Study 2011:1, Essays on Environmental Economics and Entreprenurship.
	\item Lundgren, T. and P-O. Marklund (2012) Bioenergi, klimat och Ekonomi, Miljöforskning – Formas tidning för ett hållbart samhälle, No 1, March 2012.
\end{enumerate}


\textbf{Publication list Örjan Furtenback}\\*
\textbf{Peer-reviewed articles}
\begin{enumerate}
	\item Furtenback, Ö. (2008) “Demand for waste as fuel in the swedish district heating sector: A production function approach”, Waste Management 29(1), 285-292.
\end{enumerate}

\textbf{Unpuplished working papers (available on web)}
\begin{enumerate}
	\item Furtenback, Ö. (2009) “Towards a Functional Ecol-Econ CGE Model with a Forest as Biomass Capital”, EcoMod Press, conference proceedings 2009.
	\item Furtenback, Ö.  “Dynamic CGE-model with heterogeneous forest biomass:Applications to climate policy”, CERE WP 10/2011 (2011).
\end{enumerate}

\textbf{Reports in Swedish}
\begin{enumerate}
	\item Konjunkturinstitutet (2012), ”Miljö, ekonomi och politik 2012”, Konjunkturinstitutet, Stockholm.
	\item Konjunkturinstitutet (2013), ”Miljö, ekonomi och politik 2013”, Konjunkturinstitutet, Stockholm.
\end{enumerate}

\textbf{Publication list Santosh Ram Joshi}\\*
\textbf{Peer-reviewed original articles}
\begin{enumerate}
\item Labriet M., Joshi S.R., Babonneau F., Edwards N.R., Holden P.B., Kanudia A.,  Loulou R. and Vielle M. (2013). “Worldwide impacts of climate change on energy for heating and cooling”, Mitigation and Adaptation Strategies for Global Change, DOI 10.1007/s11027-013-9522-7, Published online on 8th November 2013
\item Joshi S.R., Vielle M., Babonneau F., Edwards N.R. and Holden P.B. “Physical and economic consequences of sea-level rise: A coupled GIS and CGE analysis under uncertainties”, on revision in Environmental and Resource Economics.
\end{enumerate}
\textbf{Peer reviewed conference contribution}
\begin{enumerate}
\item Impacts of climate change on heating and cooling: a worldwide estimate from energy and macro-economic perspectives. Presented in Swiss Society of Economics and Statistics (SSES), Neuchatel, Switzerland, June 19$^{th}$ – 21$^{th}$ 2013.
\item Economic consequences of Climate Change, Fifth Integrated Assessment Modeling Consortium (IAMC) Annual Meeting, Utrecht, The Netherlands, 12$^{th}$ – 13$^{th}$ November 2012.
\item Sea level rise and its economic consequences. Presented in International Conference on Economic Modeling (ECOMOD), Seville, Spain, 4$^{th}$ – 6$^{th}$ July 2012.
\item Analyzing the sensitivity of simulated pattern of trade to functional form choices. Presented in International Conference on Economic Modeling (ECOMOD), Istanbul, Turkey,  7$^{th}$ – 10$^{th}$ July 2010.
\item Estimating an Import Demand System using the Generalized Maximum Entropy Method,
European Trade Study Group Conference (ETSG), Lausanne, Switzerland, 9$^{th}$ - 11$^{th}$ September 2010.
\end{enumerate}

\section{Illustrations (Appendix J)}
\begin{figure}%
\begin{ganttchart}[
hgrid=true,
vgrid= true,
y unit chart=0.5cm,
bar/.style={fill=gray}
]{1}{24}
\gantttitle{Months}{24} \\
\gantttitlelist{1,...,24}{1} \\
\ganttbar{Literature Review}{1}{3} \\
\ganttbar{Database}{3}{12} \\
\ganttbar{Model development}{9}{18} \\
\ganttbar{Model simulation}{16}{22} \\
\ganttbar{Result and analysis}{18}{24} \\
\ganttbar{Final Report}{21}{24}\\
\ganttbar{Workshops}{1}{1}
\ganttbar{}{8}{8}\\
\ganttbar{Meetings}{11}{11}
\ganttbar{}{16}{16}
\ganttbar{}{21}{21}
\end{ganttchart}
%http://ctan.mirrorcatalogs.com/graphics/pgf/contrib/pgfgantt/pgfgantt.pdf
\label{}%
\caption{A Gantt chart of the various tasks that will be implemented during the course of the project.}%
\end{figure}



\section{Signatures (Appendix S)}



\pagebreak
\addcontentsline{toc}{chapter}{Bibliography}
\bibliography{../RefAbstracts}
\bibliographystyle{plainnat}

\end{document}

