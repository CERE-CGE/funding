\begin{comment}
Appendix A. Research programme (25,000 chars spaces, carriage returns,  headings and footnotes included)

Including when relevant
• Specific objectives and aim of the proposed research project  
• Overview of the research area, including key references  
• Project description and a summary of the project's organization  
• A description of theory, method and performance  
• Plan for scientific deliverables  
• A description of the societal value of the research  
• Plan for communication with stakeholders/end users 
• A brief description of existing basic equipment that is relevant to the project  
• A brief description of national and international collaborations that are relevant to the project 
• Affirmation that international agreements and rules will be complied with  
• Ethical considerations  
• Gender aspects  
• For doctoral student positions, the main supervisor should be indicated 
\end{comment}

\begin{comment}
Evaluation criteria
* Criteria of Scientific Quality (most important)
	1 Research question
		• Scientific significance of the aim  
		• Originality, innovativeness and boldness of the aim, theories and/or hypotheses  
		• Potential for scientifically significant outcomes  
		• Aim in line with the call for applications 
		Clarifications: 
		• Multi- and interdisciplinary approaches should be considered advantageous when appropriate
			to the research question 
		• Gender perspectives, class, ethnicity or other social categories should be included when
			appropriate to the research question 
	2 Method and performance
		• Feasibility and adequacy of scientific methods  
		• Innovativeness and boldness of methodology  
		• Concrete and realistic work plan  
		• Concrete and realistic plan for scientific deliverables  
		• Coordination of project and research group  
		• Suitability of multi- and interdisciplinary approaches  
		• Ethical considerations
		Clarifications:  
		• Feasibility and adequacy of scientific approaches and methods should receive primary
			consideration
	3 Scientific competence
		• Quality of scientific publications  
		• Ability to carry out the project according to plan 
		• Adequate experience of project management  
		• National and international activities, including projects, networks, assignments,
			commissions of trust, and participating at or arranging workshops or conferences  
		• Strength and competitiveness of the research team  
		Clarifications:  
		• Quality of scientific publications should be assessed taking into account the
			standards within each scientific field 
		• When several researchers collaborate, assessment of scientific competence is made
			both of each researcher separately and of the research group collectively  
		• For starting mobility grants, strength and competitiveness of the research environment
			should be assessed for the home university and the host university 
* Criteria of Societal Value
	4 Societal value of research question
		• Research question addresses important societal issues and/or issues of importance for
			Formas’ sectors 
		• Project may, in a short-term or long-term perspective, contribute to sustainable
			development 
		• Awareness of stakeholder/end user needs when designing the project 
		• Aim in line with the call for applications 
		Clarifications:  
		• Awareness of stakeholder/end user needs may comprise references to e.g. directives, 
			environmental objectives and strategies, and discussions with relevant stakeholders/end
			users  
	5 Communication with stakeholders/end users
		• Description of relevant stakeholders/end users 
		• Concrete and realistic plan for communicating results with relevant stakeholders/end users  
		• Experience and ability to communicate research results with stakeholders/end users 
		Clarifications:  
		• Stakeholders/end users should be regarded in a wide sense as actors outside the 
			scientific community who can benefit from research outcomes or facilitate future 
			use in society  
		• Communication with stakeholders/end users may take different forms and be on different time
			scales depending on the research question at hand 
\end{comment}

\textbf{Purpose and aims}\\
The research project we are proposing here intends to analyze the socio-economic costs and effects of any restrictions on the capacity of the nuclear power combined with an increased demand for balancing power, due to plans of expansion of renewable weather-dependent power sources. The goal is to analyze different scenarios under which the Swedish nuclear power system might develop over the next couple of decades. To this end, an existing CGE model is targeted for enhancement on detailed description of the Swedish energy sector. This type of model, called a hybrid model, is a combination of detailed micro data on the energy sector and more aggregated data usually used in computable general equilibrium (CGE) models. This hybrid model describes the energy sector in sufficient detail enabling analysis of changes in the economic conditions of different power sources for electricity generation. This type of CGE model is currently missing for Sweden and would be a genuine contribution to the analysis of energy policy in general, but also a particularly well adapted tool for the project we are proposing here.

Energy issues are becoming increasingly important and related considerations increasingly complex. Surprisingly, energy economics is to a large extent a neglected field at Swedish universities, at least at the economics departments. CERE has launched a significant effort in this subject field. An important reason is the need for development of knowledge in CGE modeling in Sweden with focus on energy issues. Seeing the energy policy developments in a system perspective also seems increasingly important, especially in view of what has happened in the energy markets today.

Water for hydropower can be stored in reservoirs and production can hence efficiently adjust to the demand for electricity or variation of another generation, which makes it very important in the Swedish energy system. It is especially important in an energy system that is including more renewable intermittent sources, such as solar and wind power. Hydropower also has a function as base load\footnote{The minimum amount of electric power delivered or required over a given period at a constant rate.} in the Swedish electricity mix due to its large share of the Swedish electricity production mix. Another important component in the Swedish electricity production system is nuclear power. Nuclear power function as provider of base load power in Sweden but is not as important as hydropower in terms of balancing fluctuating electricity demand. A reduction in the base load capacity by reducing or decommission nuclear power while at the same time increasing intermittent sources will put a stress on hydropower in terms of both increased base load requirement and increased balancing demand at the same time. This will most certainly have impact on the scale as well as the stability of the Swedish electricity system.

There is some consensus among the leading parliamentary parties on electricity policies regarding hydropower, renewable energy, and opportunities to work with energy efficiency. The great dividing line and the main conflict lies in the perception of nuclear power. The spectra lies between those who want a quick phase out, and those who are willing to replace existing reactors.

An important question in this political setting is to what degree the Swedish economy will adapt to potential major changes in the electricity production system. A restriction in the supply of nuclear will most likely have major economic consequences in a country like Sweden with a large element of power-intensive industries dependent on stable supply of electricity. Furthermore, it is important to try to assess the spillover effects of such restrictions on the wider economy. This requires, in principle, detailed knowledge of every market in the country's economy and how these markets interact. A typical economic model to analyze this type of interaction between markets is a general equilibrium model.

CERE has recently started a group (CGE Group) to specifically analyze the Swedish and European environmental and energy policies with the help of general equilibrium modeling (CGE). CGE models' ability to describe the economy-wide effects is an important difference compared with different types of partial models. A CGE model can capture the repercussions that occur between different sectors and not only the direct impact in the sectors concerned. Compared to partial models, CGE models captures the overall socio-economic impact in a more complete and comprehensive way. The model we propose will be built upon a model (LUGEM08) constructed for specific use in the Long-Term Survey 2008, and has also been used for a number of studies of Swedish climate policy.

With the type of CGE model we propose, it will be possible to analyze the effects of policy measures targeted at individuals as well as groups of power sources in a system perspective. The tool will be an important contribution to the analysis of Swedish energy policy.

\textbf{Survey of the field}\\
Globally, nuclear energy contribute around 14\% for the world electricity generation \citep{OECD2012}. Many countries in Europe depend on nuclear energy for the electricity generation with 75\% in France , 54\% in Slovak Republic , 51\% in Belgium\footnote{http://www.iaea.org/PRIS/WorldStatistics/NuclearShareofElectricityGeneration.aspx}. About 40\% of total Swedish electricity generation in 2011 was produced from nuclear energy \citep{SEA2012}. However, the future of nuclear power is considerably uncertain in different countries due to economic, technological, environmental and political factors \citep{Joskow2012}. On one hand, nuclear power continues to generate enthusiasm based on its potential to reduce greenhouse gas emissions and comparatively cheaper to renewable energy \citep{Davis2012, Renssen2013}. On the other hand, there is growing opposition against nuclear power regarding its safety measures, handling and storage of spent fuel, and proliferation of nuclear weapons. After the Fukushima nuclear disaster, Germany and Switzerland have decided to phase out the nuclear power completely from their energy portfolio by  2022 and 2034 respectively. The position taken by Germany and Switzerland points to the potential tension between a need for climate mitigation and a desire for a society free from nuclear power \citep{Glomsrod2013}. It is quite likely that Sweden faces similar challenges in the coming future that is whether to phase out nuclear power or not with ambitious emission targets for 2030. Without a well defined capacity replacement plan, the countries could face a severe shortage of electricity supply and hence significant cost to the society. In this current context, it is important to understand what the economic implications of a nuclear phase out for Sweden, and how will this impact the electricity market and emissions in Sweden?

Several studies have shown the potential economic consequences of nuclear phase-out \citep{Bohringer2002, Nestle2012, Bretschger2012, Duscha0, Glomsrod2013, Kunsch2014}. The study by \cite{Bretschger2012} for Switzerland case showed that the nuclear phase-out can be achieved at moderate costs by investing in innovative industries and through structural change to less energy dependent economy. Moreover, if trade options within European countries are unrestricted the the short or medium term effect on electricity price will be moderate \citep{Glomsrod2013}. In a study \citep{Chen2013} of Taiwan, it is shown that it is crucial to convert industrial structure into less energy intensive if country wants to achieve non-nuclear and low carbon environment. Unlike many other studies where it is shown that domestic electricity price to decrease due to extending nuclear plant life or commissioning new nuclear plant in other countries, the study by \citep{Nestle2012} presents the evidence that this is unlikely to happen in Germany. Nuclear energy policy study on Belgium by \cite{Kunsch2014} found that a too early nuclear phase out means that Belgium economy has to rely on foreign suppliers and increased use of non-renewable and CO$_2$ emitting fossil fuels.

There are only few studies done of this in Swedish context. The study by \cite{Bergman1981} used a general equilibrium model and showed that given that the product and factor markets functions smoothly, Sweden can accommodate quite significant changes in electricity supply conditions without major changes in the main economic indicators. However, adjustment costs (reallocation of labor force, transportation cost, capital losses etc.) which are neglected in the model could give quite different results. The study by \cite{Andersson1997} used a partial equilibrium model and showed that phasing out nuclear power while restricting future CO$_2$ emissions to the 1990 level implies a significant increase in electricity prices and a substantial loss in welfare. In the Swedish Energy and Environment Policy (SEEP) model, \cite{nordhaus1997swedish} estimates the cost of nuclear phase out in Sweden (from 2000 to 2010) to 5 percent of 2010 GDP in the Swedish context. This model though, is somewhat dated and lacks a detailed description of the Swedish economy and is thus not able to capture the repercussions that occur between different sectors. Furthermore, it is important to study the current ongoing debate on nuclear phase-out in the Swedish energy and economic context with current energy and climate policy objectives.

\textbf{Project description}\\
%Summarise the project. Describe theories, methods, timetable, implementation and project organisation.
\textbf{Theories and methods}\\
As mentioned in the section Purpose and aims, a computable general equilibrium (CGE) model will be used for the purpose and aim of our research. A CGE model is theoretically consistent with microeconomic foundations. In such a model, each sector and consumer optimizes their production and consumption decision at different stages of production and utility tree within their resource, technology and budgetary constraints, respectively. These models also ensure the balance of income and expenditure of the economic agents, market clearing conditions (supply equals demand) and zero profit conditions (sum of inputs is equal to output) for each sector (commodity). Moreover, it can simulate effects not just parts of the economy that are of particular interest, but also of the effects on the entire economy. Hence, a CGE model becomes appropriate in this research when a nuclear phase out policy not only affects the electricity sector, but also will have economy-wide effects. These models are also useful for analyzing changes in sectoral output, prices, endowments and trade as well as changes in national efficiency (welfare) consequent to policy changes.

A feature of the Swedish electricity production system is that major parts as hydropower and nuclear power are subject to caps on the production capabilities at least in the short or medium term. The growth of hydropower is restricted and the restriction of the capacity of nuclear power is the target for the analysis. These types of capacity constraints are well suited to be modeled under the mixed complementarity framework \citep{raey}, which we will use for our proposed project.\\

% timetable, implementation and project organisation
% Implementation
% Data
\textbf{Implementation}\\
An initial step for the project is to compile a social accounting matrix (SAM), for Sweden, with specific focus on the energy markets. This work involves the collection of data from different sources and processing of these data so that they can consistently describe the Swedish economy in equilibrium and that the data are of sufficient detail to adequately analyze policies targeting the electricity system. This SAM will feed the model with new and complementary data.
% Model enhancement and scenarios
The existing base model will have to be modified in order to incorporate the new detailed data for the Swedish energy system and calibrated to current economic situation in order to establish a benchmark to compare with relevant policy simulations. Scenarios of different potential policy decisions will be created, and the model will need further modification in order to accommodate simulations of these scenarios.

\textbf{Project organisation}\\
% Project organisation
The research group operates within the Centre for Environmental and Resource Economics, CERE. CERE is an interdisciplinary research centre that unites Umeå University (UmU) and the Swedish University of Agricultural Sciences (SLU). Many prominent researchers are cooperating with CERE comprising an exclusive and important international network. A general aim of CERE is to provide high class research on topics pertaining to energy systems. Of special interest are the effects of climate and energy policy on energy use, which the proposed project aims to study. A well-established and experienced group of researchers can without doubt compose a notable contribution in analyzing and assessing relevant energy issues.\\*
The research group consists of:\\*
\textbf{Professor Tommy Lundgren} (principal investigator), CERE, School of Business and Economics, Umeå University. Professor Lundgren have a solid background in project management and has a number of successfully completed projects related to energy issues (funded by Energimyndigheten, Formas, Handelsbanken, Mistra, etc.).\\*
\textbf{Dr. Örjan Furtenback}, former employee of National Institute for Economic Research as a member of the CGE-group further developing the EMEC model and conducted analysis of the environmental and resource policies on the Swedish economy. As a PhD student, he developed a dynamic CGE-model with age-structured forest resource for analysis of distributional effects of incorporating forest and forest resources in reducing greenhouse gases. Dr. Furtenback will be directly involved with the implementation of the project together with Dr. Joshi.\\*
\textbf{Dr. Santosh Ram Joshi}, a PhD from NUI Galway, Ireland, and a Master from Wageningen University. He has previously worked as a modeler at the Economics and Environmental Laboratory in Switzerland. His research at CERE will in general be focused on computable general equilibrium modeling, and especially on integrating a detailed description of the energy sector into the Swedish model.\\

\textbf{Timetable}\\
% Paper writing (how many)
The research group will in cooperation write papers for peer-reviewed scientific publication and results will be presented at international meetings, workshops and conferences. We will make a special effort to put our results into popular science forums to facilitate the spread of knowledge outside the academic arena.
% Timetable
The most time consuming part of the project will be the compilation of the SAM and model modifications. The compilation of the SAM will be accomplished during the first year. Model modifications will take place during both of the years whereas creation of relevant scenarios for simulations will take place during the second year of the project.\\

\begin{ganttchart}[
hgrid=true,
vgrid= true,
y unit chart=0.5cm,
bar/.style={fill=gray}
]{1}{24}
\gantttitle{Months}{24} \\
\gantttitlelist{1,...,24}{1} \\
\ganttbar{Literature Review}{1}{3} \\
\ganttbar{Database}{3}{12} \\
\ganttbar{Model development}{9}{18} \\
\ganttbar{Model simulation}{16}{22} \\
\ganttbar{Result and analysis}{18}{24} \\
\ganttbar{Final Report}{21}{24}\\
\ganttbar{Meetings}{1}{1}
\ganttbar{}{6}{6}
\ganttbar{}{11}{11}
\ganttbar{}{16}{16}
\ganttbar{}{21}{21}
\end{ganttchart}


%http://ctan.mirrorcatalogs.com/graphics/pgf/contrib/pgfgantt/pgfgantt.pdf

\textbf{Significance}\\
With the ambitious emission targets, and the safety and disposal issue relevant for nuclear power; there is an urgent question whether Sweden should phase out, early phase out and replace its ageing nuclear power, or not. Whatever Sweden decide on all these possible options, it is important to carry out an analysis of its impact on the Swedish economy and the energy system. There are only a few attempts in the past to investigate the socio-economic consequence of nuclear phase out in Sweden. Since then, both the economic and energy systems in Sweden has evolved and changed considerably, and hence this project seeks to advance this research area within the Swedish context. This will be achieved by using a recent database, incorporating a detailed energy and electricity system, and developing relevant scenarios that are required to model the implications of a nuclear phase out. This research will add to the academic discourse on the issue of nuclear phase out and ultimately answer pertinent policy questions to give better guidance to decision and policy making.

\textbf{Preliminary results}\\
Researchers of this project have used CGE models extensively in their research area. The Global Trade Analysis Project (GTAP), a multi-regional, multi-sectoral static general equilibrium model, was used to examine and analyse the complex multi-sectoral and multilateral trade negotiations as are contained in the WTO trade policies. Similarly, GEMINI-E3 (General Equilibrium Model of International-National Interactions between Economy, energy and Environment), a recursive dynamic CGE model, is used to do macroeconomic analyse of the impacts and adaptation to climate change. This model was linked with other systems of models (climate, energy and agriculture) to get wider perspective and implications of climate change. This research was part of Framework Programme 7 project ERMITAGE (Enhancing Robustness and Model Integration for The Assessment of Global Environmental Change). In the Swedish setting the CGE-model EMEC (Environmental Medium term Economic model) of the National institute for economic research in Sweden was used to account for forest sequestration in the analysis of the Swedish vision on carbon dioxide emissions for 2050 as well as the 2030 goal for fossil fuel independence. Also a dynamic CGE-model with age-structured forest resource \citep{furtenback2011three} has been constructed by a member of the team.

\textbf{National and International collaboration}\\
Our research group will benefit from collaboration from some top scientist within the field of CGE modeling that are already affiliated with CERE. The following persons, with specific expertise and responsibility, will act as collaborating members of the project team:\\*
\textbf{Professor Christoph Böhringer}; a world leading authority on Computable General Equilibrium (CGE modeling), is a research associate at ZEW and chair at the Department of Economic Policy, University of Oldenburg, Germany. Professor Böhringer will assist in model development and issues regarding CGE modeling in an energy framework. Professor Böhringer is an important part of the CERE strategy to develop CGE modeling – particularly within the framework of our general work on energy issues. Professor Böhringer is currently working with us to develop a 57-sector GTAP version of a Sweden CGE model, which is much more aggregated regarding the energy sector. Thus, there is a beneficial cross-fertilization between this project and work already ongoing at CERE.\\*
\textbf{Dr. Martin Hill}, employee at the Ministry of Finance and Adjunct Professor at SLU and CERE, is one of Sweden's most skilled CGE specialists. As a collaborating member of the group Dr. Hill’s expertise on Swedish policy issues will be an important input when designing plausible scenarios for current and coming Swedish policies regarding nuclear and renewable power.\\*
\textbf{Dr. Badri Narayanan} is researcher at the Center for Global Trade Analysis, Department of Agricultural Economics, Purdue University, USA. Dr. Narayanan holds an extensive knowledge in data issues regarding CGE modeling and will, as a collaborating member, be an tremendous asset in the labor intense work of constructing the SAM.
